% LaTeX file for resume 
% This file uses the resume document class (res.cls)

\documentclass{res} 
\usepackage[T1]{fontenc}
\usepackage[utf8]{inputenc}
\usepackage{fancyhdr}
\usepackage{lmodern}
\usepackage[english]{babel}
\usepackage{makeidx}
\usepackage{amsfonts}
\usepackage{graphicx}
\usepackage{datetime}
\usepackage{fancyhdr}
\usepackage{lastpage}

% the margin option causes section titles to appear to the left of body text 
%\textwidth=5.2in % increase textwidth to get smaller right margin
\usepackage{helvetica} % uses helvetica postscript font (download helvetica.sty)
%\usepackage{newcent}   % uses new century schoolbook postscript font 


\pagestyle{fancy}j
\lhead{}
\chead{}
\rhead{\small{v0.4 @ \ddmmyyyydate \today}}
\lfoot{}
\cfoot{\footnotesize{ I hereby agree for processing the following personal information strictly for the purposes of job recruitment in accordance with the regulation
for the protection of personal data passed on 29.08.97r. Dz.U nr 133 poz. 883.}}
\rfoot{}

\fancypagestyle{plain}{ %
  \fancyhf{} % remove everything
  \renewcommand{\headrulewidth}{0pt} % remove lines as well
  \renewcommand{\footrulewidth}{0pt}
}

\begin{document} 
 
\name{CURRICULUM VITAE\\[12pt]} % the \\[12pt] adds a blank line after name
 
\address{{\bf Konrad Malawski} \\
	10.01.1989, Vienna \\
	mobile: +48 602 36 66 55 \\
	}
\address{
	email: konrad.malawski@project13.pl \\
    \textbf{linkedin: linkedin.com/in/konradmalawski} \\
	\textbf{github: github.com/ktoso} \\
	blog: \url{blog.project13.pl} \\
}



\begin{resume} 

\section{Professional Experience}
{\bf Senior Software Engineer} at SoftwareMill \hfill January 2012 - presently \\ 
    Focusing on \textbf{Scala backend applications} and data crunching

{\bf Trainer} at Bottega IT Solutions \hfill January 2012 - presently \\ 
	Preparing and conducting Java, Scala and other Software Craftsmanship related trainings
    
{\bf Software Developer} at Lunar Logic Polska \hfill July 2011 - December 2012 \\ 
    "JVM Passionate" - worked on large hotelier backend system \\ 
    Performed trainings for other teams \\
    Mentor of the Android Team

{\bf Trainer} at Compendium Centrum Edukacyjne \hfill November 2011 - presently \\ 
	Preparing and conducting Android, Java and Scala trainings.
    
{\bf Software Developer} at XSolve Sp. z o.o. \hfill July 2010 - June 2011\\
	GWT, Spring {AOP, MVC, DI}, Guice, GIN, Hibernate and Groovy
	Worked on backend for an online file backup system and an VM provisioning system
	Implemented multi currency payments and invoice module
    
{\bf Owner} at Project13 \hfill 2006 - presently\\
	Scala, Groovy, JEE6 or Rails Web Apps \\
	Plain J2SE, NetBeansPlatform, Griffon Desktop Apps \\
	J2ME or Android Mobile Apps 

\section{Education} 
{\bf AGH University} of Science and Technology\\ 
	Faculty of Electrical Engineering, Automatics, Computer Science and Electronics\\
	Major: Applied computer science

{\bf I Liceum Ogólnokształcące} im. B. Nowodworskiego 
	Profile: maths \& computer science

\section{Skills}
\begin{itemize}
 \item \textbf{Scala} - with real life / big project experience
 \item \textbf{Java} GWT, Hibernate, JPA2, Spring, JMS, HornetMQ, JAX-RS, Android, JVM tuning
 \item Groovy: Grails, Gradle
 \item \textbf{JavaScript}: angular.js (long before 1.0), (a bit) backbone.js, underscore.js, Jasmine
 \item Datastores: MySQL, PostgreSQL; NoSQL: \textbf{MongoDB} (experience with replicasets > 400GB)
 \item Test Driven Development (\textit{really})
 \item Practical Scrum and \textbf{Kanban} experience, also with working in distributed teams
 \item obviously: (X)HTML / CSS / XML / JSON / \textbf{ProtoBuf} / Maven / Ant / Gradle / SBT
 \item Basics of: Ruby, Python, \textbf{Go}, Hadoop / HBase
 \item VCS: very good understanding of \textbf{git}, svn
\end{itemize}

\section{Community}
\\ 
\textbf{Speaker at:}
\begin{itemize}
 \item \textbf{JFokus 2013} - Scala on Android (new talk)    \hfill 2013
 \item \textbf{ScalaCamp 2013} - Scala DSLs: Dissecting Foursquare Rogue    \hfill 2013
 \item\textbf{Java Day Riga} - Scala Intro \hfill 2012
 \item \textbf{Mobilization / KrakDroid} - Scala my Android    \hfill 2012
 \item \textbf{AGH University of Science and Technology} - Scala -- a scalable language    \hfill 2012
 \item \textbf{Java Developer Days 2011} - Effective Git    \hfill 2011
 \item \textbf{Cracow Mobi} - \textit{Dependency Injection} in Android, RoboGuice    \hfill 2011
 \item \textbf{Noc Informatyka} - Android, lightning fast intro    \hfill 2011
 \item \textbf{SFI 2010} - Git. Tak. Po prostu.    \hfill 2010
 \item \textbf{JavaCamp} - Git    \hfill 2010
\end{itemize}


\section{Trainings and Workshops}
{\bf Organizer:}
\begin{itemize}
 \item 4 day long Scala in-depth workshop \hfill 2013
 \item Multiple \textbf{JUG and GDG Meetups} \hfill 2012 - 2013
 \item Coach at \textbf{Global Day of Code Retreat} \hfill 2012 - 2013
 \item \textbf{Android workshops} for Szczecin JUG \hfill 2011
 \item \textbf{GeeCON 2011 -- 2013} \hfill 2011 - 2013
 \item \textbf{CodeRetreat}.sckrk.com \hfill 2011 -2012
 \item \textbf{Git workshop} at SFI 2012 \hfill 2010
 \item \textbf{NetBeans Platform} Certified Training (netbeans.edu.pl) \hfill 2010 \\
	http://edu.netbeans.org/courses/nbplatform-certified-training/ 
 \item and more...
\end{itemize}
{\bf Attendee:}
\begin{itemize}
 \item \textbf{JavaScript} training (incl. node.js and OOP concepts in js) (2 days) - devmeetings.pl \hfill 2011
 \item \textbf{TDD} workshop by Pragmatists (3 days) \hfill 2010
 \item \textbf{Gradle} training with Hans Dockter (Gradle CEO) \hfill 2010
 \item \textbf{Red Hat Linux} System Administration RH-131 \hfill 2009
\end{itemize}

\section{Certificates} 
{\bf Technical}
\begin{itemize}
 \item Sun \textbf{Certified Java Programmer}, Java SE 6 \hfill 2010
 \item Netbeans Platform Certified Associate \& \textbf{Netbeans Platform Certified Engineer} \hfill 2010
 \item \textbf{Red Hat Certified Technician / System
Administrator} (\#605009099327096) \hfill 2009\\
\end{itemize}
{\bf Language}
\begin{itemize}
 \item bilingual - \textbf{Native German Speaker}
 \item laureate of German Language Olympics (two times) \hfill 2004 - 2005
 \item First Certificate in English (\textbf{FCE}), grade: B \hfill 2004
\end{itemize}

\section{Other} 
\begin{itemize}
 \item \textbf{Leader:}
 \subitem \textbf{Polish Java User Group} \hfill 2010 - presently 
 \subitem \textbf{Google Developers Group Krakow} \hfill 2011 - presently
 \subitem \textbf{Kraków Scala User Group} \hfill 2013 - presently
 \item \textbf{Member:}
 \subitem \textbf{Stowarzyszenie Software Engineering Professionals Polska} \hfill 2011 - presently
 \subitem \textbf{Software Craftsmanship Kraków} \hfill 2010 - presently
 \item \textbf{Notable Open Source Projects:}
 \subitem \textbf{git-commit-id} Maven Plugin
 \subitem \textbf{ScalaWords}
 \subitem \textbf{Janbanery} - Fluent Java API for Kanbanery.com
 \item \textbf{Featured in the Oracle Java Magazine} issue March/April 2012 \\ in ,,The New Java Developers'' 
 \item \textbf{Contests:}
 \subitem Participated in Google Highly Open Participation Contest \hfill 2008
 \subitem Finalist in Motorola Diversity Contest \hfill 2007
 \item Practical knowledge about using and administering GNU/Linux systems, \\ 
       incl. basics of \textbf{Chef}
\end{itemize}

\section{Personal interests}
\begin{itemize}
 \item japanese culture and music (traditional / modern)
 \item contributing small open source projects to github.com and Maven Central
 \item distributed systems automation and configuration
 \item collecting video game consoles
 \item tennis
\end{itemize}

\end{tabular}
 
\end{resume} 
\end{document} 


