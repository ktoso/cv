% LaTeX file for resume 
% This file uses the resume document class (res.cls)

\documentclass{res} 
\usepackage[T1]{fontenc}
\usepackage[utf8]{inputenc}
\usepackage{fancyhdr}
\usepackage{lmodern}
\usepackage[polish]{babel}
\usepackage{makeidx}
\usepackage{amsfonts}
\usepackage{graphicx}
\usepackage{datetime}
\usepackage{fancyhdr}
\usepackage{lastpage}

% the margin option causes section titles to appear to the left of body text 
%\textwidth=5.2in % increase textwidth to get smaller right margin
%\usepackage{helvetica} % uses helvetica postscript font (download helvetica.sty)
%\usepackage{newcent}   % uses new century schoolbook postscript font 


\pagestyle{fancy}
\lhead{}
\chead{}
\rhead{\small{v0.1 @ \ddmmyyyydate \today}}
\lfoot{}
\cfoot{\footnotesize{Wyrażam zgodę na przetwarzanie danych osobowych zawartych w ofercie pracy dla potrzeb realizacji procesu rekrutacji (zgodnie z ustawą z dnia 29.08.1997 r. o ochronie danych osobowych Dz. U. nr 133, poz. 883)}}
\rfoot{}

\fancypagestyle{plain}{ %
  \fancyhf{} % remove everything
  \renewcommand{\headrulewidth}{0pt} % remove lines as well
  \renewcommand{\footrulewidth}{0pt}
}

\begin{document} 
 
\name{CURRICULUM VITAE\\[12pt]} % the \\[12pt] adds a blank line after name
 
\address{{\bf Konrad Malawski} \\ 
	ur. 10.01.1989 \\ 
	ul. Grenadierów 7/23 \\ 
	30-085 Kraków \\ 
	kom. 602366655 \\ 
	Narodowość: polska \\ 
%	Stan cywilny: kawaler \\ 
	}
\address{
	email: konrad.malawski@project13.pl \\
	github: github.com/ktoso \\
	portfolio: project13.pl \\
	blog: \url{blog.project13.pl} \\ 
}
 
\begin{resume} 
 
%\section{Objective} 
%Auditing/Analysis of Operations 

\section{Wykształcenie} 
{\bf Akademia Górniczo-Hutnicza} im. Stanisława Staszica w Krakowie \hfill 2008 – obecnie\\
	Wydział: Elektrotechniki, Automatyki, Informatyki i Elektroniki (EAIiE) \\
	Kierunek: Informatyka Stosowana\\ 

{\bf I Liceum Ogólnokształcące} im. B. Nowodworskiego \hfill 2005 – 2008\\
	Klasa: matematyczno - informatyczna\\ 

 

\section{Doświadczenie zawodowe}
{\bf Software Developer} w XSolve Sp. z o.o. (xsolve.pl) \hfill Lipiec 2010 – obecnie\\
	odpowiedzialny głównie za rozwój aplikacji JEE.

{\bf Freelancer} w Project13.pl \hfill 2006 – obecnie\\
	Wykonywanie stron www (PHP), animacji Flash oraz programów J2ME/Android.

\section{Przebyte szkolenia} 
{\bf Uczestnik}
\begin{itemize}
 \item 3 day long \textbf{TDD} workshop by Pragmatists \hfill 2010
 \item \textbf{Gradle} training with Hans Dockter (Gradle CEO) \hfill 2010
 \item \textbf{Red Hat Linux} System Administration RH-131 \hfill 2009
\end{itemize}
{\bf Organizator}
\begin{itemize}
 \item współorganizator cyklu spotkań \textbf{JavaCamp} \hfill 2010 - obecnie
 \item organizator \textbf{NetBeans Platform} Certified Training (netbeans.edu.pl) \hfill 2010 \\
	http://edu.netbeans.org/courses/nbplatform-certified-training/ 
 \item organizator licencji “Classroom” IntelliJ IDEA dostępnej wszystkim studentom AGH \hfill 2010
\end{itemize}

\section{Zdobyte certyfikaty} 
{\bf Informatyczne}
\begin{itemize}
 \item Sun (Oracle) \textbf{Certified Java Programmer} \hfill 2010
 \item Netbeans Platform Certified Associate & \textbf{Netbeans Platform Certified Engineer}
 \item \textbf{ Red Hat Certified Technician / System
Administrator} (RHCT/RHCSA): \hfill 2009\\
	ID: #605009099327096 https://www.redhat.com/training/certification/verify/ 
 \item ECDL \hfill 2005
\end{itemize}
{\bf Językowe}
\begin{itemize}
 \item dwukrotny laureat 'Olimpiady z języka Niemieckiego' \hfill 2004 - 2005
 \item FCE, grade: B, \hfill 2004
\end{itemize}

\section{Inne (wyróżnienia / aktywność)} 
\begin{itemize}
 \item Aktywny \textbf{członek Polish Java User Group} \hfill 2010 - obecnie
 \item Wyróżniony w konkursie Google Highly Open Participation Contest \hfill 2008
 \item Finalista w Motorolla Diversity \hfill 2007
 \item \textbf{Praktyczne} doświadczenie w pracy i administracji systemami GNU/Linuksowymi (Fedora/CentOS/Ubuntu)
 \item Prawo jazdy kat.B (czynne)
 \item Więcej linków: http://www.blog.project13.pl/index.php/\textbf{you-can-find-me-here}/
\end{itemize}

 

% Tabulate Computer Skills; p{3in} defines paragraph 3 inches wide
\section{Umiejętności}
\begin{itemize}
 \item \textbf{Java} (J2SE, J2ME, JEE, w tym podstawy JEE6)
 \subitem podstawy Hibernate, Spring, GWT
 \subitem doświadczenie z Android SDK (małe aplikacje rest-klienckie)
 \item Groovy
  \subitem Grails, podstawy Gradle (netbeans.edu.pl, GeeCON 2011 - Call for papers)
 \item PHP
  \subitem głównie Symfony Framework (1.2 - 1.4)
 \item SQL (głównie MySQL)
 \item MongoDB (w małym (choć poważnym) komercyjnym projekcie)
 \item (X)HTML / CSS / YAML / XML / JSON <- oczywiste
 \item Na poziomie 'podstawowym':
 \item JavaScript / jQuery
 \item Bash, Python
 \item Inne
  \subitem znajomość \textbf{git} i subversion
  \subitem znajomość TTD, w tym użycia Mocków (najchętniej \textbf{Mockito})
\end{itemize}

\end{tabular}

\end{resume} 
\end{document} 



